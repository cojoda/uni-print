\subsection*{What are the 5 goals of a scheduling algorithm?}
    \begin{enumerate}[label=\arabic*.]
        \item \textbf{Fairness or Waiting Time}: 
        The time a process waits in the ready queue, ensuring no process waits too long.
        \item \textbf{CPU Utilization}: 
        Maximizing CPU activity.
        \item \textbf{Response Time}: 
        The time to respond to user interactions.
        \item \textbf{Turnaround Time}: 
        The time from process submission to completion.
        \item \textbf{Throughput}: 
        The number of processes completed per time unit.
    \end{enumerate}

\subsection*{Give 2 examples where the goals are contradictory.}
    \begin{enumerate}[label=\arabic*.]
        \item \textbf{Fairness vs. Throughput}: 
        Prioritizing fairness can reduce throughput, as frequent context switching to ensure no process waits too long increases overhead, lowering the number of completed processes.
        \item \textbf{Response Time vs. CPU Utilization}: 
        Optimizing for response time can underutilize the CPU, as prioritizing quick user input may leave the CPU idle while waiting for more input.
    \end{enumerate}